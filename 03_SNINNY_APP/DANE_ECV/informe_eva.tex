% Options for packages loaded elsewhere
\PassOptionsToPackage{unicode}{hyperref}
\PassOptionsToPackage{hyphens}{url}
%
\documentclass[
  11pt,
]{article}
\usepackage{amsmath,amssymb}
\usepackage{iftex}
\ifPDFTeX
  \usepackage[T1]{fontenc}
  \usepackage[utf8]{inputenc}
  \usepackage{textcomp} % provide euro and other symbols
\else % if luatex or xetex
  \usepackage{unicode-math} % this also loads fontspec
  \defaultfontfeatures{Scale=MatchLowercase}
  \defaultfontfeatures[\rmfamily]{Ligatures=TeX,Scale=1}
\fi
\usepackage{lmodern}
\ifPDFTeX\else
  % xetex/luatex font selection
\fi
% Use upquote if available, for straight quotes in verbatim environments
\IfFileExists{upquote.sty}{\usepackage{upquote}}{}
\IfFileExists{microtype.sty}{% use microtype if available
  \usepackage[]{microtype}
  \UseMicrotypeSet[protrusion]{basicmath} % disable protrusion for tt fonts
}{}
\makeatletter
\@ifundefined{KOMAClassName}{% if non-KOMA class
  \IfFileExists{parskip.sty}{%
    \usepackage{parskip}
  }{% else
    \setlength{\parindent}{0pt}
    \setlength{\parskip}{6pt plus 2pt minus 1pt}}
}{% if KOMA class
  \KOMAoptions{parskip=half}}
\makeatother
\usepackage{xcolor}
\usepackage[margin=1in]{geometry}
\usepackage{color}
\usepackage{fancyvrb}
\newcommand{\VerbBar}{|}
\newcommand{\VERB}{\Verb[commandchars=\\\{\}]}
\DefineVerbatimEnvironment{Highlighting}{Verbatim}{commandchars=\\\{\}}
% Add ',fontsize=\small' for more characters per line
\usepackage{framed}
\definecolor{shadecolor}{RGB}{248,248,248}
\newenvironment{Shaded}{\begin{snugshade}}{\end{snugshade}}
\newcommand{\AlertTok}[1]{\textcolor[rgb]{0.94,0.16,0.16}{#1}}
\newcommand{\AnnotationTok}[1]{\textcolor[rgb]{0.56,0.35,0.01}{\textbf{\textit{#1}}}}
\newcommand{\AttributeTok}[1]{\textcolor[rgb]{0.13,0.29,0.53}{#1}}
\newcommand{\BaseNTok}[1]{\textcolor[rgb]{0.00,0.00,0.81}{#1}}
\newcommand{\BuiltInTok}[1]{#1}
\newcommand{\CharTok}[1]{\textcolor[rgb]{0.31,0.60,0.02}{#1}}
\newcommand{\CommentTok}[1]{\textcolor[rgb]{0.56,0.35,0.01}{\textit{#1}}}
\newcommand{\CommentVarTok}[1]{\textcolor[rgb]{0.56,0.35,0.01}{\textbf{\textit{#1}}}}
\newcommand{\ConstantTok}[1]{\textcolor[rgb]{0.56,0.35,0.01}{#1}}
\newcommand{\ControlFlowTok}[1]{\textcolor[rgb]{0.13,0.29,0.53}{\textbf{#1}}}
\newcommand{\DataTypeTok}[1]{\textcolor[rgb]{0.13,0.29,0.53}{#1}}
\newcommand{\DecValTok}[1]{\textcolor[rgb]{0.00,0.00,0.81}{#1}}
\newcommand{\DocumentationTok}[1]{\textcolor[rgb]{0.56,0.35,0.01}{\textbf{\textit{#1}}}}
\newcommand{\ErrorTok}[1]{\textcolor[rgb]{0.64,0.00,0.00}{\textbf{#1}}}
\newcommand{\ExtensionTok}[1]{#1}
\newcommand{\FloatTok}[1]{\textcolor[rgb]{0.00,0.00,0.81}{#1}}
\newcommand{\FunctionTok}[1]{\textcolor[rgb]{0.13,0.29,0.53}{\textbf{#1}}}
\newcommand{\ImportTok}[1]{#1}
\newcommand{\InformationTok}[1]{\textcolor[rgb]{0.56,0.35,0.01}{\textbf{\textit{#1}}}}
\newcommand{\KeywordTok}[1]{\textcolor[rgb]{0.13,0.29,0.53}{\textbf{#1}}}
\newcommand{\NormalTok}[1]{#1}
\newcommand{\OperatorTok}[1]{\textcolor[rgb]{0.81,0.36,0.00}{\textbf{#1}}}
\newcommand{\OtherTok}[1]{\textcolor[rgb]{0.56,0.35,0.01}{#1}}
\newcommand{\PreprocessorTok}[1]{\textcolor[rgb]{0.56,0.35,0.01}{\textit{#1}}}
\newcommand{\RegionMarkerTok}[1]{#1}
\newcommand{\SpecialCharTok}[1]{\textcolor[rgb]{0.81,0.36,0.00}{\textbf{#1}}}
\newcommand{\SpecialStringTok}[1]{\textcolor[rgb]{0.31,0.60,0.02}{#1}}
\newcommand{\StringTok}[1]{\textcolor[rgb]{0.31,0.60,0.02}{#1}}
\newcommand{\VariableTok}[1]{\textcolor[rgb]{0.00,0.00,0.00}{#1}}
\newcommand{\VerbatimStringTok}[1]{\textcolor[rgb]{0.31,0.60,0.02}{#1}}
\newcommand{\WarningTok}[1]{\textcolor[rgb]{0.56,0.35,0.01}{\textbf{\textit{#1}}}}
\usepackage{graphicx}
\makeatletter
\def\maxwidth{\ifdim\Gin@nat@width>\linewidth\linewidth\else\Gin@nat@width\fi}
\def\maxheight{\ifdim\Gin@nat@height>\textheight\textheight\else\Gin@nat@height\fi}
\makeatother
% Scale images if necessary, so that they will not overflow the page
% margins by default, and it is still possible to overwrite the defaults
% using explicit options in \includegraphics[width, height, ...]{}
\setkeys{Gin}{width=\maxwidth,height=\maxheight,keepaspectratio}
% Set default figure placement to htbp
\makeatletter
\def\fps@figure{htbp}
\makeatother
\setlength{\emergencystretch}{3em} % prevent overfull lines
\providecommand{\tightlist}{%
  \setlength{\itemsep}{0pt}\setlength{\parskip}{0pt}}
\setcounter{secnumdepth}{5}
\usepackage{booktabs}
\usepackage{longtable}
\usepackage{array}
\usepackage{multirow}
\usepackage{wrapfig}
\usepackage{float}
\usepackage{colortbl}
\usepackage{pdflscape}
\usepackage{tabu}
\usepackage{threeparttable}
\usepackage{threeparttablex}
\usepackage[normalem]{ulem}
\usepackage{makecell}
\usepackage{xcolor}
\ifLuaTeX
  \usepackage{selnolig}  % disable illegal ligatures
\fi
\usepackage{bookmark}
\IfFileExists{xurl.sty}{\usepackage{xurl}}{} % add URL line breaks if available
\urlstyle{same}
\hypersetup{
  pdftitle={Informe EVA --- Explorador territorial y aglomeración},
  pdfauthor={Observatorio EVA},
  hidelinks,
  pdfcreator={LaTeX via pandoc}}

\title{Informe EVA --- Explorador territorial y aglomeración}
\author{Observatorio EVA}
\date{15 oct 2025}

\begin{document}
\maketitle

{
\setcounter{tocdepth}{2}
\tableofcontents
}
\section{1. Portada y resumen
ejecutivo}\label{portada-y-resumen-ejecutivo}

\textbf{Indicador principal:} Producción (Ton)\\
\textbf{Ámbito geográfico:} País\\
\textbf{Cultivo:} Todos\\
\textbf{Año focal:}

\begin{verbatim}
## **Resumen**: No hay serie suficiente para construir un resumen cuantitativo.
\end{verbatim}

\begin{quote}
\textbf{Lectura recomendada:} este informe prioriza la interpretación
operativa: niveles, cambios recientes y jerarquías territoriales; no
implica causalidad, sino \textbf{diagnóstico descriptivo} de los datos
cargados en la app.
\end{quote}

\section{2. Contexto y alcance}\label{contexto-y-alcance}

Este documento presenta un panorama del indicador \textbf{Producción
(Ton)} en el ámbito nacional.\\
El año focal de mapas y ranking es **** (según la selección del tab),
mientras que la \textbf{serie temporal} resume la trayectoria anual del
agregado bajo los filtros vigentes (país/departamento/municipio y
cultivo).

\textbf{Interpretación del indicador:}

\begin{itemize}
\tightlist
\item
  \emph{Área sembrada / cosechada}: suma del área reportada.\\
\item
  \emph{Producción}: suma de toneladas.\\
\item
  \emph{Rendimiento (Ton/Ha)}: \textbf{Σ Producción / Σ Área cosechada}
  bajo el subconjunto filtrado (evita promedios simples sesgados).
\end{itemize}

\begin{quote}
Cuando el filtro ``Departamento'' = \textbf{Todos}, las estadísticas
representan \textbf{agregados nacionales}. Si se selecciona un
departamento, los mapas del tab ``Explorador'' cambian a nivel municipal
dentro de dicho departamento.
\end{quote}

\section{3. Serie temporal del
indicador}\label{serie-temporal-del-indicador}

\begin{verbatim}
## No hay datos de serie temporal para graficar.
\end{verbatim}

\textbf{Lectura rápida de la serie:}

\begin{itemize}
\tightlist
\item
  \textbf{Dirección reciente:} no disponible.\\
\item
  \textbf{Picos/vasos:} no disponible.\\
\item
  \textbf{Sugerencia metodológica:} contrastar con clima, precios o
  asistencia técnica, según el caso (fuera del alcance de este informe).
\end{itemize}

\section{4. Mapa (según tab) y jerarquías
territoriales}\label{mapa-seguxfan-tab-y-jerarquuxedas-territoriales}

\begin{verbatim}
## Mapa no proporcionado por la aplicación (parámetro `map_png` o `clusters_png`).
\end{verbatim}

\begin{quote}
\textbf{Nota:} En ``Clusters'' los colores reflejan el tipo LISA con
umbral \textbf{p ≤ 0.50`} (asociación espacial, \textbf{no} causalidad).
\end{quote}

\section{5. Ranking Top-10 (nivel
municipal)}\label{ranking-top-10-nivel-municipal}

\begin{verbatim}
## No hay datos de ranking para graficar.
\end{verbatim}

\textbf{Ideas de interpretación:}

\begin{itemize}
\tightlist
\item
  Identificar \textbf{picos} (municipios muy por encima de la mediana
  departamental/nacional).
\item
  Ver si el liderazgo proviene de \textbf{cartera amplia} de cultivos o
  de un \textbf{cultivo dominante} (ver Tabla de detalle).
\end{itemize}

\section{6. Tabla de detalle
(exportable)}\label{tabla-de-detalle-exportable}

\begin{verbatim}
## No hay tabla disponible.
\end{verbatim}

\section{7. Aglomeración (solo si se adjunta mapa
LISA)}\label{aglomeraciuxf3n-solo-si-se-adjunta-mapa-lisa}

\section{8. Metodología y
consideraciones}\label{metodologuxeda-y-consideraciones}

\begin{itemize}
\tightlist
\item
  \textbf{Fuentes y filtros:} subconjunto definido en la app
  (Departamento, Municipio y Cultivo). Mapas/ranking con año ****; la
  serie usa todo el rango disponible.\\
\item
  \textbf{Transformaciones clave:}

  \begin{itemize}
  \tightlist
  \item
    \emph{Rendimiento (Ton/Ha)} = \textbf{Σ Producción / Σ Área
    cosechada}.\\
  \item
    \emph{Áreas y producción}: sumatorias simples según filtros.\\
  \end{itemize}
\item
  \textbf{Aglomeración (LISA):} clústeres (Alto--Alto, Bajo--Bajo,
  Alto--Bajo, Bajo--Alto, No significativo) vía contigüidad ``queen'',
  \emph{p} 0.50.\\
\item
  \textbf{Limitaciones:} resultados \textbf{descriptivos}; para
  causalidad, contrastar con clima, precios, infraestructura, crédito,
  etc.
\end{itemize}

\section{9. Recomendaciones
operativas}\label{recomendaciones-operativas}

\begin{itemize}
\tightlist
\item
  \textbf{Monitoreo}: replicar en cortes trimestrales/semestrales.\\
\item
  \textbf{Priorización territorial}: cruzar Top-10 con clústeres LISA
  para focalizar intervenciones.\\
\item
  \textbf{Validez}: revisar atípicos y consistencia con registros
  administrativos y encuestas sectoriales.
\end{itemize}

\textbf{Fin del informe.}

\subsection{\texorpdfstring{2) Pegado rápido de los
\texttt{downloadHandler} (sustituye SOLO el \texttt{content\ =\ \{\}} en
tu
app)}{2) Pegado rápido de los downloadHandler (sustituye SOLO el content = \{\} en tu app)}}\label{pegado-ruxe1pido-de-los-downloadhandler-sustituye-solo-el-content-en-tu-app}

2.1. En el botón \textbf{PDF --- Tab Explorador}

\begin{Shaded}
\begin{Highlighting}[]
\CommentTok{\# PDF — Tab Explorador (Rmd)}
\NormalTok{output}\SpecialCharTok{$}\NormalTok{dl\_pdf\_expl }\OtherTok{\textless{}{-}} \FunctionTok{downloadHandler}\NormalTok{(}
  \AttributeTok{filename =} \ControlFlowTok{function}\NormalTok{() }\FunctionTok{paste0}\NormalTok{(}\StringTok{"EVA\_informe\_"}\NormalTok{, input}\SpecialCharTok{$}\NormalTok{f\_indicador, }\StringTok{"\_"}\NormalTok{, }\FunctionTok{Sys.Date}\NormalTok{(), }\StringTok{".pdf"}\NormalTok{),}
  \AttributeTok{content =} \ControlFlowTok{function}\NormalTok{(file) \{}
    \CommentTok{\# 1) Exportar PNG del mapa simple actual}
\NormalTok{    widget  }\OtherTok{\textless{}{-}} \FunctionTok{map\_widget\_simple}\NormalTok{()}
\NormalTok{    tmp\_html }\OtherTok{\textless{}{-}} \FunctionTok{tempfile}\NormalTok{(}\AttributeTok{fileext =} \StringTok{".html"}\NormalTok{)}
\NormalTok{    tmp\_png  }\OtherTok{\textless{}{-}} \FunctionTok{tempfile}\NormalTok{(}\AttributeTok{fileext =} \StringTok{".png"}\NormalTok{)}
\NormalTok{    htmlwidgets}\SpecialCharTok{::}\FunctionTok{saveWidget}\NormalTok{(widget, tmp\_html, }\AttributeTok{selfcontained =} \ConstantTok{TRUE}\NormalTok{)}
\NormalTok{    webshot2}\SpecialCharTok{::}\FunctionTok{webshot}\NormalTok{(tmp\_html, }\AttributeTok{file =}\NormalTok{ tmp\_png, }\AttributeTok{vwidth =} \DecValTok{1200}\NormalTok{, }\AttributeTok{vheight =} \DecValTok{800}\NormalTok{, }\AttributeTok{zoom =} \DecValTok{2}\NormalTok{)}

    \CommentTok{\# 2) Parámetros al Rmd}
\NormalTok{    params }\OtherTok{\textless{}{-}} \FunctionTok{list}\NormalTok{(}
      \AttributeTok{indicador\_label =} \FunctionTok{as.character}\NormalTok{(}\FunctionTok{isolate}\NormalTok{(}\FunctionTok{indicador\_label}\NormalTok{())),}
      \AttributeTok{depto    =} \ControlFlowTok{if}\NormalTok{ (}\FunctionTok{is.null}\NormalTok{(}\FunctionTok{isolate}\NormalTok{(input}\SpecialCharTok{$}\NormalTok{f\_depto))) }\StringTok{"Todos"} \ControlFlowTok{else} \FunctionTok{as.character}\NormalTok{(}\FunctionTok{isolate}\NormalTok{(input}\SpecialCharTok{$}\NormalTok{f\_depto)),}
      \AttributeTok{mpio     =} \ControlFlowTok{if}\NormalTok{ (}\FunctionTok{is.null}\NormalTok{(}\FunctionTok{isolate}\NormalTok{(input}\SpecialCharTok{$}\NormalTok{f\_mpio)))  }\StringTok{"Todos"} \ControlFlowTok{else} \FunctionTok{as.character}\NormalTok{(}\FunctionTok{isolate}\NormalTok{(input}\SpecialCharTok{$}\NormalTok{f\_mpio)),}
      \AttributeTok{cultivo  =} \ControlFlowTok{if}\NormalTok{ (}\FunctionTok{is.null}\NormalTok{(}\FunctionTok{isolate}\NormalTok{(input}\SpecialCharTok{$}\NormalTok{f\_cultivo))) }\StringTok{"Todos"} \ControlFlowTok{else} \FunctionTok{as.character}\NormalTok{(}\FunctionTok{isolate}\NormalTok{(input}\SpecialCharTok{$}\NormalTok{f\_cultivo)),}
      \AttributeTok{anio\_sel =} \ControlFlowTok{if}\NormalTok{ (}\FunctionTok{is.null}\NormalTok{(}\FunctionTok{isolate}\NormalTok{(input}\SpecialCharTok{$}\NormalTok{f\_anio))) }\StringTok{""} \ControlFlowTok{else} \FunctionTok{as.character}\NormalTok{(}\FunctionTok{isolate}\NormalTok{(input}\SpecialCharTok{$}\NormalTok{f\_anio)),}
      \AttributeTok{serie    =} \FunctionTok{isolate}\NormalTok{(}\FunctionTok{series\_data}\NormalTok{()),}
      \AttributeTok{ranking  =} \FunctionTok{isolate}\NormalTok{(}\FunctionTok{ranking\_data}\NormalTok{()),}
      \AttributeTok{tabla    =} \FunctionTok{isolate}\NormalTok{(}\FunctionTok{tabla\_export}\NormalTok{()),}
      \AttributeTok{map\_png  =}\NormalTok{ tmp\_png,}
      \AttributeTok{clusters\_png =} \ConstantTok{NULL}\NormalTok{,}
      \AttributeTok{lisa\_p =} \FloatTok{0.50}
\NormalTok{    )}

    \CommentTok{\# 3) Renderizar Rmd {-}\textgreater{} PDF}
\NormalTok{    rmarkdown}\SpecialCharTok{::}\FunctionTok{render}\NormalTok{(}
      \AttributeTok{input  =} \StringTok{"informe\_eva.Rmd"}\NormalTok{,}
      \AttributeTok{output\_file =} \StringTok{"informe\_expl.pdf"}\NormalTok{,}
      \AttributeTok{params =}\NormalTok{ params,}
      \AttributeTok{envir  =} \FunctionTok{new.env}\NormalTok{(}\AttributeTok{parent =} \FunctionTok{globalenv}\NormalTok{())}
\NormalTok{    )}
    \FunctionTok{file.copy}\NormalTok{(}\StringTok{"informe\_expl.pdf"}\NormalTok{, file, }\AttributeTok{overwrite =} \ConstantTok{TRUE}\NormalTok{)}
\NormalTok{  \}}
\NormalTok{)}
\end{Highlighting}
\end{Shaded}

2.2. En el botón \textbf{PDF --- Tab Clusters}

\begin{Shaded}
\begin{Highlighting}[]
\CommentTok{\# PDF — Tab Clusters (Rmd)}
\NormalTok{output}\SpecialCharTok{$}\NormalTok{dl\_pdf\_clus }\OtherTok{\textless{}{-}} \FunctionTok{downloadHandler}\NormalTok{(}
  \AttributeTok{filename =} \ControlFlowTok{function}\NormalTok{() }\FunctionTok{paste0}\NormalTok{(}\StringTok{"EVA\_informe\_"}\NormalTok{, input}\SpecialCharTok{$}\NormalTok{clus\_indicador, }\StringTok{"\_"}\NormalTok{, }\FunctionTok{Sys.Date}\NormalTok{(), }\StringTok{"\_clusters.pdf"}\NormalTok{),}
  \AttributeTok{content =} \ControlFlowTok{function}\NormalTok{(file) \{}
    \CommentTok{\# 1) Exportar PNG del mapa LISA simple}
\NormalTok{    widget  }\OtherTok{\textless{}{-}} \FunctionTok{map\_clusters\_simple}\NormalTok{()}
\NormalTok{    tmp\_html }\OtherTok{\textless{}{-}} \FunctionTok{tempfile}\NormalTok{(}\AttributeTok{fileext =} \StringTok{".html"}\NormalTok{)}
\NormalTok{    tmp\_png  }\OtherTok{\textless{}{-}} \FunctionTok{tempfile}\NormalTok{(}\AttributeTok{fileext =} \StringTok{".png"}\NormalTok{)}
\NormalTok{    htmlwidgets}\SpecialCharTok{::}\FunctionTok{saveWidget}\NormalTok{(widget, tmp\_html, }\AttributeTok{selfcontained =} \ConstantTok{TRUE}\NormalTok{)}
\NormalTok{    webshot2}\SpecialCharTok{::}\FunctionTok{webshot}\NormalTok{(tmp\_html, }\AttributeTok{file =}\NormalTok{ tmp\_png, }\AttributeTok{vwidth =} \DecValTok{1200}\NormalTok{, }\AttributeTok{vheight =} \DecValTok{800}\NormalTok{, }\AttributeTok{zoom =} \DecValTok{2}\NormalTok{)}

    \CommentTok{\# 2) Parámetros al Rmd}
\NormalTok{    params }\OtherTok{\textless{}{-}} \FunctionTok{list}\NormalTok{(}
      \AttributeTok{indicador\_label =} \FunctionTok{as.character}\NormalTok{(}\FunctionTok{isolate}\NormalTok{(}\FunctionTok{clus\_indicador\_label}\NormalTok{())),}
      \AttributeTok{depto    =} \FunctionTok{as.character}\NormalTok{(}\FunctionTok{isolate}\NormalTok{(input}\SpecialCharTok{$}\NormalTok{clus\_depto)),}
      \AttributeTok{mpio     =} \StringTok{"Todos"}\NormalTok{,}
      \AttributeTok{cultivo  =} \ControlFlowTok{if}\NormalTok{ (}\FunctionTok{is.null}\NormalTok{(}\FunctionTok{isolate}\NormalTok{(input}\SpecialCharTok{$}\NormalTok{clus\_cultivo))) }\StringTok{"Todos"} \ControlFlowTok{else} \FunctionTok{as.character}\NormalTok{(}\FunctionTok{isolate}\NormalTok{(input}\SpecialCharTok{$}\NormalTok{clus\_cultivo)),}
      \AttributeTok{anio\_sel =} \FunctionTok{as.character}\NormalTok{(}\FunctionTok{isolate}\NormalTok{(input}\SpecialCharTok{$}\NormalTok{clus\_anio)),}
      \AttributeTok{serie    =} \FunctionTok{isolate}\NormalTok{(}\FunctionTok{series\_data}\NormalTok{()),}
      \AttributeTok{ranking  =} \FunctionTok{isolate}\NormalTok{(}\FunctionTok{ranking\_data}\NormalTok{()),}
      \AttributeTok{tabla    =} \FunctionTok{isolate}\NormalTok{(}\FunctionTok{tabla\_export}\NormalTok{()),}
      \AttributeTok{map\_png  =} \ConstantTok{NULL}\NormalTok{,}
      \AttributeTok{clusters\_png =}\NormalTok{ tmp\_png,}
      \AttributeTok{lisa\_p =} \FloatTok{0.50}
\NormalTok{    )}

    \CommentTok{\# 3) Renderizar Rmd {-}\textgreater{} PDF}
\NormalTok{    rmarkdown}\SpecialCharTok{::}\FunctionTok{render}\NormalTok{(}
      \AttributeTok{input  =} \StringTok{"informe\_eva.Rmd"}\NormalTok{,}
      \AttributeTok{output\_file =} \StringTok{"informe\_clus.pdf"}\NormalTok{,}
      \AttributeTok{params =}\NormalTok{ params,}
      \AttributeTok{envir  =} \FunctionTok{new.env}\NormalTok{(}\AttributeTok{parent =} \FunctionTok{globalenv}\NormalTok{())}
\NormalTok{    )}
    \FunctionTok{file.copy}\NormalTok{(}\StringTok{"informe\_clus.pdf"}\NormalTok{, file, }\AttributeTok{overwrite =} \ConstantTok{TRUE}\NormalTok{)}
\NormalTok{  \}}
\NormalTok{)}
\end{Highlighting}
\end{Shaded}


\end{document}
